% !TEX root = ../thesis.tex
\chapter{Conclusion}
\label{ch:conclusion}
% **************************** Define Graphics Path **************************
\ifpdf
    \graphicspath{{Chapter6/Figs/Raster/}{Chapter6/Figs/PDF/}{Chapter6/Figs/}}
\else
    \graphicspath{{Chapter6/Figs/Vector/}{Chapter6/Figs/}}
\fi

\section{Introduction}

In Chapter \ref{ch:approach}, a crowd monitoring framework using emotion analysis of social media has been proposed. The framework was implemented and evaluated in a case study with a past sporting event where a stampede occurred in Chapter \ref{ch:eval}. The evaluation showed that the using the tweets collected about the event, our Emotion Analysis and Rule Based Reasoning was able to identify the correct crowd types in a timely manner. This chapter will summarise the outcome and the contribution of our research.

This chapter is structured as follow. Firstly, the research contribution will be discussed, followed by the future research including enhancement to the existing framework and possible directions for further study.

\section{Research Contribution}
The contribution of our research is that the proposed framework have addressed the gaps identified in the literature review. Firstly, by leveraging the potential of social media analysis, our proposed framework employed social media as the information source for crowd monitoring. Secondly, the framework has incorporated \textcite{Berlonghi1995}'s model, a standard crowd model, into crowd monitoring in order to enable the consistency and interoperability between different monitoring systems and to provide a standard and uniform classification of different crowds. Finally, the evaluation of the framework showed the great potential of the method using emotion analysis of social media to monitor the crowd in mass gathering events. These three contributions are also related back to the objectives of the framework that have been defined in Chapter \ref{ch:approach}. The following sections will further discuss each contribution.

\subsection{Social Media as Information Source}
Using social media as the information source for crowd monitoring has some advantages compared with other sources such as CCTV or mobile sensors. There is no requirement for hardware or software installed at the venue or on the mobile devices of participants. Despite the potential of using social media, the literature review points out that very limited work has been done to apply social media analysis into crowd monitoring to support emergency management in mass gathering.

This research proposed a crowd monitoring framework that utilised social media as the information source. In our framework, tweets about an event were collected as the raw context data about the condition of the crowd. The emotion analysis was applied to raw level context data to transform it into high level context data that was the emotional states of the crowd. These emotions were used in the crowd type classification.

\subsection{Crowd Model and Emotion - Crowd Type Mapping Model}
Understanding the need of a using a standard crowd model to classify different types of crowds, our framework has adopted the eleven crowd types defined by \textcite{Berlonghi1995}. Although the Berlonghi's model is commonly adopted in the literature of emergency management, this work only described the eleven crowd types from such points as the purposes of gathering or the activities in the crowd \parencite{Zeitz2009}. Hence, it is very difficult to use those features in an automatic crowd type classification without additional characteristics that can be measured.

Realising the impact of emotion on the behaviour of the crowd \parencite{Kornblum2011}, our research introduced emotional states as the features to classify different types of crowds. Our research employed the four emotions: \textit{anger}, \textit{fear}, \textit{happiness} and \textit{sadness} adopted from \textcite{ekman1971constants}'s basic emotions as the Emotion Model in the crowd monitoring framework. Based on a wide range of researches on social behaviour and psychology, an Emotion - Crowd Type Mapping Model has been also proposed to map between a crowd type with its associated emotions. Using the mapping model, our research categorised eleven crowd types into five distinctive groups based on their common motivating emotions. To infer the emotional states of the crowd, our research relied on the Emotion Analysis, which will be discussed in the next section.

\subsection{Emotion Analysis of Social Media in Crowd Monitoring}
In our framework, the Emotion Analysis extracted the emotion from the tweets based on a Bag-of-Words approach and the association scoring using the NRC Hashtag Emotion Corpus \parencite{mohammad2014using} as the labelled corpus. The emotion rates of \textit{anger}, \textit{fear}, \textit{happiness} and \textit{sadness} were selected as the measures of the emotional states of the crowd. The change in the emotion rate of an emotion can be detected using moving average and z-score, as proposed in Chapter \ref{ch:eval}. When the z-score exceeded a certain threshold, the level of that emotion in the crowd is considered as \textit{high}. The Rule Based Reasoning contained a set of rules that were able to identify the group of crowd type from the levels of emotions in the crowd. 

To evaluation the proposed framework, a case study with historical data collected from a boxing match where a stampede occurred was performed. The result showed that the framework was able to detect the correct crowd types in a timely manner. This has confirmed the great potential of applying the proposed emotion analysis of social media in a real-time crowd monitoring to support emergency management in mass gathering.

\section{Future Research}
There exist several issues regarding our proposed framework, such as the capability to support real-time monitoring or the performance of the Emotion Analysis. Essentially, these issues might suggest the direction for future research to further improve the crowd monitoring for emergency management in mass gathering.

\subsection{Real-time Support of the Framework}
In our evaluation, a method using case study and historical data was used. The case study has confirmed both the correctness and timeliness of the detection produced by our proposed framework. However, the case study was not able to evaluate the framework in term of performance, especially real-time performance. This suggests future researches to work on the real-time support of the framework.

In a real-time crowd monitoring, the performance of the Emotion Analysis must be able to analyse a large number of tweets collected in real-time. In our implementation for the experiment, the tweets were processed in a batch by a single thread which might not be applicable in the real-time analysis. Such techniques as online data stream processing are suggested to boost the performance of the emotion extraction process.

\subsection{Improvement of Emotion Analysis}
In our experiment, the Emotion Analysis was able to label 84.67\% of the collected tweets with an emotion. A number of tweets could not be labelled because of following reasons. A large proportion of those tweets did not contain text message but URL and emoticons which were discarded during the tokenization. Another reason is that because the NRC Hashtag Emotion Corpus had only 16,862 words, an unlabelled tweet might contain the words which were not included in the training set, thus the association scoring could not be computed. Therefore, the emotion analysis can be further improved by extending the labelled corpus.

Secondly, our emotion model considers the four most basic emotions extracted from \textcite{ekman1971constants}'s work based on a recent research by \textcite{Jack2014}. These four emotions: \textit{anger}, \textit{fear}, \textit{happiness} and \textit{sadness} are also considered to have the clear influence on the crowd behaviours. However, the emotion in a tweet might be more complicated than the basic emotions. Therefore, it can be suggested that a future study extends the emotion model to consider more emotions and investigates their association with the crowd behaviour.

The Emotion Analysis in our proposed framework detects the emotion from a tweet by a simple method that extracts the dominant emotion in that tweet using the emotional weight vectors of words and the summative vector of the tweet. The advantage of this method is simplicity, however, future work might improve the emotion extraction by applying more advanced classifiers, such as SVM.

Finally, our Bag-of-Words approach is a very simple approach that only considers the appearance of the words in a tweet while ignoring their ordering and grammar. A semantic approach might be desirable in a future research to take into account the meanings of the words.

\subsection{Integrating Mobile Sensing}
Using emotion as the feature of the crowd, in our research, five distinct groups of crowd types can be identified by their similarity in term motivating emotions. However, it is currently not possible to further identify the exact crowd types in the same group without obtaining additional information. As mentioned in the Chapter \ref{ch:approach}, this additional information can be retrieved by integrating more information sources and analysis techniques into the framework. 

Utilising mobile sensors and mobile sensing techniques might be considered as one of the potential enhancement to the framework in the future. For example, using activity recognition over the data collected from accelerometers built on participants’ mobile devices, the crowd movement can be inferred and used as an additional crowd feature in crowd type classification. The crowd movement can be used to distinguish the crowd types within the same group, for example, the expressive crowd and the participatory crowd under Group 2 motivated by \textit{happiness}.

The advantage of our framework is the modularisation of the components and the use of a standard crowd model, it is possible to extend and integrate with more component. Mobile sensing can be integrated into the Context Data component as another information source. The activity recognition technique can be implemented as another analysis component.