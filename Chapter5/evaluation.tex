% !TEX root = ../thesis.tex
\chapter{Evaluation}
\label{ch:eval}
% **************************** Define Graphics Path **************************
\ifpdf
    \graphicspath{{Chapter5/Figs/Raster/}{Chapter5/Figs/PDF/}{Chapter5/Figs/}}
\else
    \graphicspath{{Chapter5/Figs/Vector/}{Chapter5/Figs/}}
\fi

\section{Introduction}

\section{An Evaluation using Historical Data}
In our proposed framework, the accuracy of the crowd type classification relies on the Emotion Analysis and the Rule Based Reasoning. As presented in the Chapter \ref{ch:approach}, the Emotion analysis detects the emotion from the tweets collected about an gathering event and computes the emotion distribution of the crowd. From this emotion distribution, the Rule Based Reasoning identifies the correct crowd types using a set of defined rules. 

In order to evaluate the accuracy of the approach, we decided to conduct an experiment using historical data because of following reasons. Firstly, although the number of mass gathering events around the world are considerably high, crowd accidents do not occur very often. Therefore, a possibly large number of experiments with real events must be done until a potential detection is found, which makes our evaluation to be both time and effort consuming. Secondly, even when a possible incident has been detected, it is still difficult to tell what exactly has happened in the crowd until the investigation announces the findings and it usually takes weeks or months. With the smaller accidents with very few injuries, there is a chance that an investigation would never be made. This would make cause more challenge to our evaluation. 

On the other hand, using historical data it is possible to select an event in the past, in which a crowd related accident eventually occurred and a known crowd type was identified. Our experiment will then focus on the data retrieval and comparison between our finding and the fact. Secondly, it is also a drawback of our approach that is the inability to support other language than English because our Bag-of-Words were constructed in English. Using historical data allows us to narrow down on the events in English speaking country where it is easier to collect a sufficient number of English tweet for the analysis.

In spite of having such clear advantages over the real-time experiment, there is a certain disadvantage with the evaluation strategy using historical data. The performance of the framework regarding real-time support is not tested, such as whether the data collection technique can gather sufficient context data to support a real-time analysis or whether the response of the analysis is timely enough. Hence it is required for further experiments to be done in the future in order to fully evaluate the framework.

\subsection{The Event}
The first step of the experiment is to select a suitable mass gathering event that matches the criteria:
\begin{inparaenum}[i)]
\item the mass gathering took place in an English speaking country;
\item a crowd accident occurred and a known crowd type was identified;
\item the event was a recent event that preferably happened after 2012
\end{inparaenum}. The reason for the last criteria is that Twitter was created in 2006 and its traffic only started booming since 2012 \footnote{http://www.internetlivestats.com/twitter-statistics/}. Therefore, it is not possible to gather tweets about an event that happened too far in the past. For these reasons, we primarily looked for recent sporting events as these events tend to involve a large number of participants and high arousal emotions. Finally, we selected the boxing match in USA between Floyd Mayweather and Marcos Maidana on May 4th 2014, where a stampede occurred after the fight \footnote{http://www.latimes.com/sports/sportsnow/la-sp-sn-mayweather-stampede-20140504-story.html}.

The match was held at the Grand Garden Arena of the MGM Grand Hotel in Las Vegas, USA with more than 16000 people attended \footnote{http://www.usatoday.com/story/sports/boxing/2014/05/04/floyd-mayweather-marcos-maidana-stampede/8700763/}. The stampede happened around 10PM local time (UTC-7) or 3PM on 5th May Melbourne time (UTC+10). According to the official statement, when the fans were leaving the arena, the stampede was triggered by a loud bang of a temporary wall falling over that was mistaken as a sound of a gunshot. It caused a panic and people started rushing to the exits. More than 50 people were injured as being trampled and crushed in the stampede. The layout of the arena was later criticized for having only two exits and the pathway was too narrow which has caused a bottleneck \footnote{http://sports.yahoo.com/news/bob-arum--stampede-after-mayweather-maidana-was--an-accident-ready-to-happen-004057893-boxing.html}. This event appears to satisfy our conditions defined above. The next section will describe the data collection process to gather tweets about this incident.

\subsection{Data Collection}
There are several methods to access to Twitter's data. The most common way is to access via Twitter's official Search API \footnote{https://dev.twitter.com/rest/public/search}. However, according to the documentation, the API is limited to return only tweets from the past week, hence it is not usable to retrieve historical data in our experiment. Another method is access the data from the 3rd party providers, such as Topsy \footnote{http://topsy.com} which allows to search for tweets back in the past using keywords. This website is also capable to refine the search results by date and language, yet it does not support filtering by geo-location or places. 

Our data collection requires to gather tweets related to a relatively short event. Unlike searching tweets about a celebrity or a long-term event, it is very difficult to provide the exact keywords to locate the tweets about such short term event. As we know the venue of the event, it is easier to filter for tweets by location information. Therefore, a more flexible method employing the Twitter Advanced Search and web crawling technique, was introduced in our experiment.

\subsubsection{Twitter Advanced Search}
Twitter Advanced Search is in fact a built-in functionality on Twitter's homepage \footnote{https://twitter.com/search-advanced}, which allows to search for any tweet on Twitter. It supports searching with keywords, date range, places and most importantly, it does not have any restriction in term of historical data \footnote{https://support.twitter.com/articles/71577-using-advanced-search}. This functionality is also freely accessible through the Twitter website without any authentication.

On the other hand, since the Advanced Search is designed for the user who accesses Twitter via a browser, the search result is displayed in a web page in HTML format. It does not support exporting the data to any format that is programmatically readable. Another technical challenge is that the search results are wrapped inside a scrolling pane, which requires the scrolling action from the user to fetch more data. 

In conclusion, although Twitter Advanced Search is very a potential source to collect historical tweets, it is not designed to be programmatically accessed. However, because our experiment required a larger number of tweets to perform analysis, it was impractical to manually perform the data collection. Therefore, the web crawling technique was incorporated to automate the process.

\subsubsection{Crawling the Twitter Advanced Search}
In order to programmatically access Twitter Advanced Search, PhantomJS and CasperJS were utilized in our data collection process. PhantomJS \footnote{http://phantomjs.org} is headless scripted browser which is used to automate web page interaction for testing purposes. CasperJS \footnote{http://casperjs.org} is an open source Javascript library written for PhantomJS to enhance various tasks including the DOM manipulation. The complete process consisting of visiting Twitter page, submitting the search queries, scrolling down the result list and extracting the needed information from the DOM were written in Javascript as step by step instructions for the headless browser to operate. In other words, the headless browser imitated the human interaction on a normal browser and automatically stored the extracted tweets into a file.

As the stampede occurred in the evening 4th May 2014, the two parameters of the Advanced Search \textit{since} and \text{until} were set as ``2014-05-04'' and ``2014-05-05'' respectively which was long enough to cover the whole event. Because our analysis only supports English, the corresponding parameter \textit{lang} was set as ``eng'' to filter out tweets written in other languages. Because the venue of the boxing match was MGM Grand Hotel, the parameter \textit{near} was set to the coordinates of the hotel \footnote{36.102552, -115.169569} with \textit{within} set to 5 miles. The search engine returned all tweets that were either geotagged or checked in with a place within the radius of specified area.

It was noticeable in the returned tweets that there were a large number of tweets containing the mention \textit{@MGMGrand} or the hashtag \textit{#MGMGrand}. They suggested to be a possible keyword and hashtag that can be used to alternatively search for tweets at the venue. Therefore, we ran another crawler with similar parameters but searching for ``MGMGrand'' instead of specifying the location.

Finally, two sources were combined and a 10 hour period of data around the time when the accident happened was extracted. This data was used to evaluate the capability of the framework to detect the stampede. The total number of tweets within the chosen period was 6351 tweets.

\subsubsection{Thresholds Calculation}
The threshold is used to identify a high level of an emotion, suggesting potentially abnormal crowd behaviour. As mentioned in Chapter \ref{ch:approach}, the thresholds are different for each emotion and their values are application specific.

\subsubsection{The Experimental Dataset}

\subsection{Analysis}

\subsection{Discussion}

\section{Conclusion}