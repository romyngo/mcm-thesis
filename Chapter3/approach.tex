% !TEX root = ../thesis.tex
\chapter{Approach}
\label{ch:approach}
% **************************** Define Graphics Path **************************
\ifpdf
\graphicspath{{Chapter3/Figs/Raster/}{Chapter3/Figs/PDF/}{Chapter3/Figs/}}
\else
\graphicspath{{Chapter3/Figs/Vector/}{Chapter3/Figs/}}
\fi

\section{Introduction}
\label{sec:approachIntro}
Despite the importance of crowd monitoring in mass gathering events, the literature review in the previous chapter is able to identify several gaps in the research, especially in crowd modelling. Most state-of-art crowd monitoring techniques do not incorporate a typology to distinguish different types of crowd. In other words, no explicit type of crowd is defined in those approaches.

say something more about crowd modelling

conclude with the need of a crowd monitoring framework

\section{An Overview of the Crowd Monitoring Framework}

The literature review in Chapter \ref{ch:litReview} has summarised that most of the state-of-art crowd monitoring approaches are focusing on using computer vision to automate the analysis of CCTV system. The literature review has also noted several limitations of the computer vision technique, such as the effect of obstacles and low lighting condition. With the rapid development of mobile computing and the increasing popularity of mobile devices, mobile sensing seems to be a more potential technique to collect contextual data for crowd monitoring, for example GPS receivers and accelerometers. Context data refers to all the knowledge that is related to a crowd. Context data can be at raw level such as the acceleration and rotational forces captured by accelerometer integrated in a participant's mobile phone, or at high level such as the current activity that user is performing.

Apart from the ``hard sensors'', the literature review also highlights that another type of information source known as the ``soft sensor'' can be used to detect the occurrence of stampede in a crowd which is the social media \citep{Ramesh2014}. The use of social media analysis for crowd monitoring has been introduced in a related work by \citet{DelirHaghighi2013}. It has the great advantage of feasibility as it only relies on the software and no additional hardware is required to be installed. Despite this strength, our literature review shows that very limited works have been done to utilize social media to support emergency management.

Secondly, another finding from the literature review is that emotions are one of the key factors that motivate the collective behaviours \citep{Kornblum2011, jasper2011emotions}, which in our context are the crowd behaviours. For that reason, it is essential for a crowd monitoring approach to consider the influence of emotions in the crowd. Although human emotion is an abstract concept and very difficult to measure with ``hard sensors'', it is possible to capture the emotion of a person from the verbal expressions, such as text \citep{alm2005emotions} or speech \citep{sobin1999emotion}. This is where social media further proves its advantage over the ``hard sensors''. By applying analysis on the social media, we can capture the emotions in a crowd, thus enabling the inference of the crowd's condition. For that reason, in this project, we would like to propose a crowd monitoring framework that employs the emotion analysis of social media to support emergency management in mass gatherings. 

Finally, the literature review also reveals another gap in the research that is the limitation of the crowd modelling employed in crowd monitoring approaches. Most existing crowd monitoring techniques do not base on any crowd model that can distinguish between different crowd types and identify the exact crowd type is happening without the interpretation of human. By exploring broader into other disciplines, our literature review points out several notable works on crowd modelling and classification in the police literature and public safely science. Among those works, \citet{Berlonghi1995}'s work stands out as the most commonly adopted model by the emergency management bureaus worldwide. Our proposed crowd monitoring framework is established on the \citet{Berlonghi1995}'s model which consists of eleven different crowd types. Yet, a systematic approach to classify one crowd into those types is challenging because \citet{Berlonghi1995}'s model only describes the crowd types and there is no attribute defined in the existing model to differentiate between them. Therefore, in our approach, we would like to propose a mapping model that connects the emotion model with crowd model, enabling the the fuzzy classification of crowd type by emotions. The reason fuzzy logic is utilized in our approach is because of its capability of representing multiple truths rather than the absolute true or false in classical logic. Since many crowd types can exist in one large crowd \citep{Berlonghi1995}, fuzzy logic is able to calculate the degree of membership of each crowd type in a particular crowd.

In conclusion, our proposed approach will address the gaps that are identified in the previous chapters by: 
\begin{inparaenum}[i)]
\item incorporating social media as the information source for context data;
\item capturing the emotions of a crowd by analysing the context data;
\item identifying the type of a crowd by applying fuzzy rule based inference to the crowd emotions
\end{inparaenum}. These objectives can be integrated together in a complete process as illustrated by Figure \ref{fig:processOverview}. Social media is firstly probed to get the context data at the raw level. This raw data is then processed in an emotion analysis to extract high level context data, that is the emotional states of the crowd. These emotional states are in turn used as the attributes or features to classify a crowd into a specific type.

\begin{figure}[htb!] 
\centering    
\includegraphics[width=1.0\textwidth]{ProcessOverview}
\caption{A process to identify a crowd type from social media}
\label{fig:processOverview}
\end{figure}

From functional point of view, the process can be implemented into a framework. Figure \ref{fig:frameworkOverview} shows our framework consisting of following components:
\begin{inparaenum}[i)]
\item Context data;
\item Emotion analysis;
\item Emotion model;
\item Crowd model;
\item Emotion - Crowd type mapping model;
\item Rule based reasoning
\end{inparaenum}. Each component will be discussed further in the sections below.

\begin{figure}[htb!] 
\centering    
\includegraphics[width=1.0\textwidth]{FrameworkOverview}
\caption{An overview of the crowd monitoring framework using emotion analysis of social media}
\label{fig:frameworkOverview}
\end{figure}

\section{Context data}
The first component of the framework is context data component. In our proposed framework, contextual data is gathered from the social media as the information source. Social media is the generic term referring to a wide range of Internet based tools that enable an user to create and share information with each other. These tools include social network services such as Facebook and Twitter, blogs, Internet forums and channels which has marked the beginning of Web 2.0. Unlike other traditional media such as newspaper or television, the content on social media is created by the users, or also known as user-generated content. In different social media, user-generated content can be in various formats such as text, image or video. Because of the fact that the content is being generated by the users themselves, social media is an effective source to obtain the knowledge about the users. In our domain and application of crowd monitoring, the kind of knowledge that is important to capture is the emotional state of the participants in the crowd. 

In our research, we will focus on the use of social media, or Twitter in particular, as the information source for context data, because of following reasons. Firstly, among social media, Twitter is the most commonly used in researches because of its large volume of users and its public APIs which make the data highly accessible. The user-generated content is mostly text-based and in the form of a short message which has no more than 140 characters called tweet. Because of this length limit, a tweet is usually very simple in term of the meaning and each tweet focuses on expressing the idea of the author on one particular topic. This fact makes the semantic analysis of tweets more feasible than other content, for example, a long blog article.

A tweet might also contain information beyond the text. A significant number of tweets are geo-tagged, which means they are attached with location information when they are posted to Twitter. This location information can be in the form of exact coordinates if the user posts the tweet from a GPS enabled mobile phone. Alternatively, this location might also be a relative location such as point of interests or town if the user checks in or attaches a place with the tweet.

Another useful information that might exist in a tweet is hashtags. Hashtags are created by the users, and usually they are put in a tweet to refer to the topic mentioned in the tweet. In some mass gathering events such as the Australia Open or music festivals, there are hashtags created for these particular events so that a user can put the hashtags into his tweet when mentioning about the events. Similarly to the geo-location, this information can also be used to filter for tweets belonging to a specific event.

Finally, using Twitter as the information source for crowd monitoring has some certain advantages. Firstly, the data can be collected in real-time, enabling the real-time monitoring which is an essential operation in emergency management because of the dynamic nature of a crowd that it can change from a calm type to an aggressive type \citep{Berlonghi1995}. Secondly, it does not require any pre-installed hardware or software on the participants, which increases the feasibility of our proposed framework.

\section{Emotion Model}

From the literature review, a crowd type can be described as a form of collective behaviour of a group of people in close proximity. Both the Convergence Theory proposed by Allport, Millar and Dollard and the Emergent Norm Theory by Turner and Killian support the idea that a crowd behaviour is formed by people with a common motivation \citep{mcphail1991myth}. In our framework, we emphasize on the emotional factor as the common motivation of the behaviour in the crowd. Different emotions can consequently lead to different crowd behaviours. For example, according to Lofland, a panic exodus from burning theatre is motivated by the fear whereas a race riot is aroused by the anger \citep{Kornblum2011}. Therefore, an emotion model is essential to distinguish different emotions of the people in the crowd.

To represent the emotions of the participants in a crowd, our framework adapts \citet{Plutchik1980}'s wheel of emotion as the baseline for human basic emotions. Figure \ref{fig:emotionModel} illustrates the eight different basic emotions identified by Plutchik, consisting of anger, anticipation, joy, trust, fear, surprise, sadness and disgust. Each basic emotion has an opposite emotion, with which together forms a pair of contradictory emotions, such as sadness/joy or fear/anger. According to Plutchik's theory of emotion, other human emotions are either being a form of the lower and higher intensity of the eight emotions, or being a complex emotion composed of multiple basic emotions. For example, aggressiveness is a combination of anger and anticipation or remorse is a combination of sadness and disgust.

\begin{figure}[htb!]
\centering    
\includegraphics[width=1.0\textwidth]{PlutchikModel}
\caption{The Plutchik's wheel of emotions}
\label{fig:emotionModel}
\end{figure}

As can be seen from Figure \ref{fig:emotionModel}, the emotion wheel has eight wings representing eight basic human emotions. In each wing, the intensity of an emotion is increasing as it gets closer to the centre of the wheel. Adjacent emotions can formulate another advanced emotion as mentioned in the earlier examples. However in our approach, the emotion model only concentrates on the most basic eight emotions, and will not explore further into the intensity nor the combination of these emotions.

\section{Emotion Analysis of Social Media}

A recent analysis on the content on Twitter reveals that the users are using Twitter for a variety of purposes, which can be grouped in two main types: updating on themselves or sharing information \citep{java2007we}. Regardless of the content, in both cases a tweet often conveys the author's emotional status \citep{bollen2009modeling}. By capturing the emotion expressed in the text, it is possible to infer the emotional state of the author. In our framework, as can be seen from Figure \ref{fig:processOverview} a tweet can be considered as a raw context data collected from the information source whereas the emotion in the tweet is the high-level context data to be used in later crowd type classification. This transformation from the raw-level to the high-level context data is performed by the emotion analysis component.

There have been several approaches to extract the emotion from Twitter in the literature \citep{roberts2012empatweet, bollen2009modeling, mohammad2012emotional, mohammad2014using}, which have great influence on our method. Our proposed method is based on a combination of Bag-of-Word model, weighting and a simple voting scheme, which will be discussed in the next sections.

\subsection{Bag-of-Word Model}
Bag-Of-Word model is a simple representation of a document which only considers at the frequency of appearance of the words in a document while disregarding the grammar and the order of the words. In our research using Twitter as the corpus, a document is a single tweet and a word is an uni-gram word building up the tweet. The process that split a document into a sequence of words is usually called tokenizer. In our experiment which will be mentioned in Chapter \ref{ch:eval}, we applies the Stanford NLP Tokenizer to tokenize a tweet into words and also to filter out data that is not in our interest, such as URLs and emoticons. 

Figure showing Bag-Of-Word and tokenizer.

In the Bag-Of-Word representation, the number of occurrence of a word in the tweet contributes to its significance in the tweet. For example, if a word appears twice in the tweet, it will have a doubled weight compared with words with single appearance, thus being more significant. In our context of emotion analysis, the weight of a word represents its association with a specific emotion from the list of eight basic emotions mentioned above. Therefore, a word will have eight weight scores for each emotions: anger, anticipation, joy, trust, fear, surprise, sadness and disgust, which can be illustrated by a proposed ``emotional weight vector'' with eight dimensions corresponding to the eight emotions. 

Figure showing the weight vector with 8 dimensions.

In conclusion, using Bag-Of-Word approach, we can consider that a tweet consists of a series of words, each of which has its association with the eight emotions described by a emotional weight vector. The next section will cover the method to to calculate the ``emotional weight vectors'' of words and to identify the dominating emotion in a tweet.

\subsection{Emotion Word Corpus and Emotional Weight Vector}
Our emotion analysis can be considered as a supervised classification of emotion for a tweet. The analysis requires a collection of words to be gathered and the ``emotional weight vector'' of each word in the collection must be calculated accordingly. Such collection is often referred as the corpus. In order for the calculation of ``emotional weight vector'', a corpus of tweets that have been labelled with a correct emotion for each tweet is necessary.

Emotion analysis from text is not a new area in the field of natural language processing. There have been a number of works regarding the construction of text-emotion corpus from various sources such as mails \citep{mohammad2011tracking} and books \citep{mohammad2011once}, newspaper headlines \citep{strapparava2008learning}. Those proposed corpora can be integrated into our framework. However, the language used in a tweet can be significantly different from other text content because of the difference in the usages that a tweet is limited to 140 characters and the users mostly use Twitter to update on themselves or share information. Therefore, only the corpora that are built from Twitter will be of interest to our the framework.

EmpaTweet proposed by \citet{roberts2012empatweet} is a corpus on Twitter that is annotated with seven emotions: anger, disgust, fear, joy, love, sadness and surprise. The tweets are collected in 14 topics which are chosen in a manner that each topic is expected to evoke a particular emotion in the tweets. Another notable corpus is the NRC Hashtag Emotion corpus, in which tweets are collected by emotion-word hashtags corresponding to Ekman's six emotions: \#anger, \#disgust, \#fear, \#happy, \#sadness and \#surprise and their synonyms\citep{mohammad2012emotional}. This approach appears to be more robust as it relies on the self-labelled emotions by the authors rather than the topic-based emotion assumption like EmpaTweet. It is also verified that the self-annotated emotions eventually match the emotions annotated by the judges. Because of this robustness, the NRC Hashtag Emotion corpus is selected as the emotion word corpus in our emotion analysis. The corpus has n tweets and m words.

From the NRC Hashtag Emotion corpus, \citet{mohammad2012emotional} also calculates the association of a word with each of the six emotions, which is called Strength of Association (SoA) based on the frequency of occurrence of this word in tweets labelled with a specific emotion. The score is from 0 indicating no association to infinity indicating maximum association. The result is a word emotion lexicon which list all words in the corpus with theirs SoA with the six emotions, which is made available for research purposes. In the NRC Hashtag Emotion Lexicon version 2.0, the SoA is calculated to Plutchik's eight emotions. This lexicon can be employed as the baseline for our calculation of the ``emotional weight vector''. The eight emotions is equivalent to the eight dimensions of the vector. For each emotion, the SoA score represents the the magnitude of the vector in the corresponding dimension.

From the weight vectors of the words constructing a tweet, the dominating emotion depicted in the tweet can be determined using a voting system which will be discussed in the following section.

\subsection{Voting System and Dominant Emotion}
Because a word can have the different weights in the eight emotions, a tweet consisting of a set of words also consequently possesses its association with the eight emotions. It can be notated as a vector that is summation of the weight vectors of the words constructing the tweet.

Formula showing sum of vector here

Although a tweet might be associated with all eight emotions, a specific emotion is expected to dominate the whole tweet. In order to extract the dominant emotion in a tweet, we propose following voting system that can interpret the vector as follow. An ``emotional weight vector'' can be considered as a vote cast by a word, that indicates how strongly it agrees with each emotion in list. Similarly, the summative vector can be interpreted as the overall result of the voting, which illustrates how close a tweet is to each emotion. The dominant emotion can be then determined by the highest vote from the voters. In other words, the dimension which has the largest magnitude will be chosen as the dominant emotion in the tweet.

\subsubsection{Emotions in the Crowd}
As illustrated by Figure \ref{fig:processOverview}, the tweets collected in the first stage of the process provides the context at the raw-level about a crowd in a mass gathering. By performing the emotion analysis, the dominant emotion in each tweet can be extracted. This emotion in fact represents the emotion of an individual in the crowd rather than the crowd as a whole. The relationship between individual emotion and group emotion has been discussed in several researches. \citet{barsade1998group} suggest that group emotion can arise and felt by individual member, while at the same time, the group emotion can be shaped by the compositional effects of individual emotions. This theory confirms the possibility to represent the crowd emotion as whole from the individuals' emotion.

Because of the fact that people with different emotional state can exist in one crowd, we introduce the fuzziness to the representation of emotions in a crowd. The fuzziness allows a crowd to have multiple emotions with a different degree of membership toward each emotion. To measure the degree of membership of a crowd to a specific emotion from the individual emotion of the participants, we propose following method. The degree of membership of a crowd's emotion to emotion \(e_i\) at a given time \(t_j\) is calculated by below formula.
\[
	d(e_i) = \frac{n(e_i)}{N}
\]
where \(d(e_1)_{t_j}\) is the degree of membership toward emotion \(e_1\), \(N\) is the total number of data sampled between \(t_{j-1}\) and \(t_j\) and \(n(e_i)\) is the number of data classified with emotion \(e_i\) in \(N\).

This degree of membership \(d(e_i)\) can have any value from 0 to 1, indicating the how close the emotion of the crowd is to \(e_i\) at time \(t_j\). The calculation of the degrees of membership is performed repetitively with an interval which allows us to detect the change of emotional state of the crowd. The value of t will have influence on the sensitivity of the detection and depend on the sampling rate.

\section{Crowd Model}

The ultimate objective of our crowd monitoring framework is to identify the type of a crowd. Therefore, a crowd model that can describe and differentiate different crowd types is required. From the emergency management point of view, the literature review highlights \citet{Berlonghi1995}'s model as one of the most significant works on crowd types. As mentioned in Chapter \ref{ch:litReview}, \citet{Berlonghi1995} has identified eleven different types of crowd in mass gathering. Each crowd type is described by the movement, participation and behaviour \citep{Zeitz2009}. \citet{Berlonghi1995}'s definition has been widely adopted as the guideline by the emergency bureaus in Australia and United States of America. However, relying only on the description, it is difficult to make distinct of different crowd types without human observation and judgement on the spot. Hence, there is a need to introduce attributes or features to the existing model for a systematic approach to perform the classification of crowd type.

In our proposal framework, we introduce \citet{Plutchik1980}'s eight emotions as the features to \citet{Berlonghi1995}'s crowd model. The next section focuses on how the mapping between each crowd type and its associated emotion is constructed.

\section{Emotion - Crowd Type Mapping Model}
In the previous sections, we have discussed the possibility to detect the emotions in a crowd. In our approach, anger, anticipation, joy, trust, fear, surprise, sadness and disgust are used as the eight features of a crowd type. This section will explore further into the relationship between each crowd type and each emotion.

Table \ref{table:mappingEmotionCrowdType} presents our proposal for the mapping between each crowd type and eight basic emotions.
\begin{table}

\caption{Mapping between crowd types and emotions}
\label{table:mappingEmotionCrowdType}
\begin{tabular}{|p{2cm}|p{1.2cm}|p{1.2cm}|p{1.2cm}|p{1.2cm}|p{1.2cm}|p{1.2cm}|p{1.2cm}|p{1.2cm}|}
\hline
\textbf{Crowd type}	& \textbf{Anger}	& \textbf{Anticipation}	& \textbf{Joy} 	& \textbf{Trust}	& \textbf{Fear}	& \textbf{Surprise}	& \textbf{Sadness}	& \textbf{Disgust}	\\
\hline
Ambulatory			& low 				& medium				& medium		& medium			& low 			& low 				& medium			& low 		\\
\hline
Limited movement	& medium			& medium				& low 			& low 				& low 			& low 				& medium			& medium	\\
\hline
Spectator			& low 				& medium				& medium		& medium 			& low 			& medium			& low 				& low 		\\
\hline
Participatory		& low 				& medium				& medium		& medium			& low 			& low 				& low 				& low 		\\
\hline
Aggressive			& medium			& low 					& low 			& low 				& low 			& low 				& low 				& medium	\\
\hline
Demonstrator		& medium			& low 					& low 			& low 				& low 			& low 				& medium			& medium	\\
\hline
Escaping			& low 				& low 					& low 			& low 				& medium		& medium			& low 				& low 		\\
\hline
Dense				& medium			& low 					& low 			& low 				& medium		& medium			& low 				& medium	\\
\hline
Rushing				& medium			& low 					& low 			& low 				& low 			& low 				& low 				& medium	\\
\hline
Violent				& medium			& low 					& low 			& low 				& low 			& low 				& low 				& medium	\\
\hline
\end{tabular}
\end{table}

\section{Rule Based Reasoning}

\subsection{Fuzzifier}
We convert the numerical values of the frequency of appearance of an emotion into linguistic label: low, medium and high by applying following rules

if the value is below the low threshold of that emotion then the label is low
if the value is above the low threshold and below the high threshold then the label is medium
if the value is above the high threshold then the label is high

\subsection{Rule Repository}

list 11 rules here for 11 crowd types

\subsection{Inference Engine}

Fuzzy rule here and the mathematical formula here

\subsection{Output Processor}

the output is a vector of each crowd and its weight

\section{Conclusion}