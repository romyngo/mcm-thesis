\chapter{Approach}

% **************************** Define Graphics Path **************************
\ifpdf
    \graphicspath{{Chapter3/Figs/Raster/}{Chapter3/Figs/PDF/}{Chapter3/Figs/}}
\else
    \graphicspath{{Chapter3/Figs/Vector/}{Chapter3/Figs/}}
\fi

\section{Introduction}
From the literature review, we can notice that Berlonghi's crowd typology is the well adopted in the context of public safety management. However, the definition of each crowd type proposed by Berlonghi is the purpose of the gathering, rather than the description of the crowd, thus making it difficult to classify a crowd. Therefore, in this chapter, we would like to propose more attributes to be applied into the current model and propose an approach to classify a crowd into those evelen types. 

\section{Crowd Model}

\subsection{Refinement of Berlonghi's model}
Berlonghi's model is chosen as the crowd model in our approach because the eleven types defined in the model are sufficient to support almost all crowd types defined by related works. Table below illustrates the overlap between Berlonghi's crowd model and other model mentioned in the previous chapter.

\begin{center}
	\begin{longtable}{|p{2cm}|p{2cm}|p{2cm}|p{2cm}|p{2cm}|p{2cm}|p{2cm}|p{2cm}|p{2cm}|}
	\caption{}
	\label{} \\
	\hline
	\textbf{Crowd type} & \textbf{Crowd type} & \textbf{Source} & \textbf{Definition} & \textbf{Example} & \textbf{Level of density} & \textbf{Level of movement} & \textbf{Crowd activities} & \textbf{Motivating emotion} \\
	\hline
	\end{longtable}
\end{center}

Table above also presents several attributes which can be used to distinguish different crowd types. Each attribute will be further discussed in below sections.

\subsubsection{Level of density}

\subsubsection{Level of movement}

\subsubsection{Crowd activities}

\subsubsection{Motivating emotions}

\subsection{Emotion in the crowd}

\section{Crowd Monitoring using Emotion Analysis of Social Media}

\subsection{Social Media Analysis}

\subsubsection{Twitter as the Context data}

\subsubsection{Twitter Analysis}

\subsection{Emotion Analysis}

\subsubsection{The Word-Emotion Lexicon}

\subsubsection{The use of Voting System for Emotion Classification}

\subsection{Realtime Crowd Monitoring}

\subsubsection{Mapping emotion into crowd types}

\subsubsection{Realtime detection of dangerous crowd types}

\section{Conclusion}