%*******************************************************************************
%*********************************** First Chapter *****************************
%*******************************************************************************
% !TEX root = ../thesis.tex
\chapter{Introduction}  %Title of the First Chapter
\label{ch:intro}
\ifpdf
    \graphicspath{{Chapter1/Figs/Raster/}{Chapter1/Figs/PDF/}{Chapter1/Figs/}}
\else
    \graphicspath{{Chapter1/Figs/Vector/}{Chapter1/Figs/}}
\fi

\section{Introduction}

To maintain the safety in mass gatherings, there is a need for event organizers and emergency services to quickly detect emerging or potentially critical situations in the crowd \citep{Wirz2012} . Hence, crowd monitoring plays a very important role for emergency management in such large scale events. Unfortunately, the current methods used in the monitoring operations such as patrol, close-circuit television (CCTV) observation or drones are costly, time and labour consuming and affected by human error, such as the loss of concentration \citep{Davies1995}. Much effort has been made to introduce computer-based approaches to automate the monitoring and analysis of the crowd. Recent advances in mobile computing also enable a wide range of rich context data sources and sensing paradigms to detect the emergency situations in the crowd, such as participatory sensing by GPS location trace \citep{Wirz2012} and Bluetooth identifier \citep{Weppner2013} collected from the sensors integrated on participants' mobile phones, or analysing social media to identify the dangerous crowd types \citep{DelirHaghighi2013}.

Although these novel approaches were able to detect the critical situation in the crowd by applying different quantitative reasoning, most of them lacked the use of a crowd model which can enable the consistency and interoperability between different monitoring systems. Another benefit of using a crowd model is that it provides a standard and uniform classification of different crowd types. Therefore, there is clearly a need for the new researches to incorporate a standard crowd model into a crowd monitoring approach.

The state-or-the-art in crowd monitoring shows the potential of using different information sources for context data, such as CCTV cameras \citep{Davies1995}, sensors integrated on mobile phones \citep{Wirz2013} and social media \citep{DelirHaghighi2013}. Although each source has its own strength and limitation when used in a monitoring system, employing social media as the information source has the certain advantages over other sources. Firstly, it is not affected by the operating conditions, such as the light condition like in computer vision based approaches using CCTV cameras. Secondly, it does not require the installation of any hardware at the venue nor software on participants' device which might potentially raise the privacy concern. Despite these strength, very limited work has been done using social media for the purpose of crowd monitoring.

This project aims to propose a crowd monitoring framework employs social media as the information source for contextual data. Secondly, from existing studies on crowd modelling, this project also aims to adopt and refine to construct a standard crowd model for emergency management in mass gatherings.

\section{Background}

\subsection{Emergency Management in Mass Gathering}

A mass gathering is defined as an event attended by a large crowd of spectators and participants. According to \citep{Arbon2007}, the number of participants in a mass gathering event can be more than 25000 people. The types of event can also vary from religious events, sporting events or concerts. Because of the large number of participant and high density as well as the the impact of psychological factors such as crowd mood \citep{Arbon2004}, there is a high risk of emergency situations to occur in mass gathering. In fact, the crowd accidents in such events eventually generate a higher rate of injury and illness than the general population statistics \citep{Arbon2007}. As a result, lots of attention is being given to improve the emergency management in mass gathering, including the planning and the provisioning of medical and security services. 

\begin{table}[!htbp]
	\caption{Three phases of Emergency Management in Mass Gathering}
	\label{table:phaseOfEm}
	\centering
	\begin{tabular}{|p{3cm}|p{3cm}|p{3cm}|}
		\hline
		\multicolumn{3}{|c|}{\textbf{Emergency Management phases}} \\ \hline \hline
		\textbf{Pre-event} & \textbf{During event} & \textbf{Post event} \\	\hline
		Planning & Monitoring  & Auditing  \\
		Preparation & Communication & Evaluation \\
		& Response & Debriefing \\
		\hline
	\end{tabular}
\end{table}

Emergency management in mass gathering events can be phased into three stages: 
\begin{inparaenum}[i)]
	\item pre-event;
	\item during the event;
	\item post-event
\end{inparaenum} as described in Table \ref{table:phaseOfEm} \citep{DelirHaghighi2013}. Among these stages, the provisioning of emergency services during the event is the most challenging as it requires real-time interaction and communication between staffs as well as real-time decision making. Therefore, monitoring plays the an important role for these operations because most of the decision making during this stage rely on the intelligence obtained from this monitoring. 

A literature review conducted by \citet{Soomaroo2012} on crowd disasters at mass gathering events discovered that a significant number of case report blamed a poor response time of the emergency services. Therefore, in order to prevent future incidents, more effort should be made on monitoring of the crowd during the course of an event. 

\subsection{An Overview of Crowd Monitoring}

According to \citet{Berlonghi1995}, in order to assure the safety of a mass gathering event, two types of operations are involved that are the crowd management and the crowd control. Crowd management includes all measures taken to facilitate the movement and enjoyment of participants. It can be understood as a proactive effort, in contrast to the crowd control which is a reactive effort taken when the crowd is out of control or when an incident occurs. Crowd monitoring can be considered as the bridge between the crowd management and the crowd control operations because it helps to identify the potential critical situations in the crowd so that appropriate response can be provisioned in a timely manner.

To monitor the crowd in mass gathering event, at the moment following techniques are usually practised: patrol and CCTV observation or even helicopter/drone surveillance for a large and open venue. However, more than half of the worldwide incidents occurred in developing countries in Asia and Africa \citep{BurkleJr2011} where emergency management personnel is often less prepared and both human resources and equipments are very limited. Moreover, as highlighted by \citet{Davies1995}, there are certain drawbacks of using CCTV monitoring as well. Firstly, this task is highly time consuming and labour intensive because of the large number of cameras and recordings captured by the CCTV system. Secondly, the human observers are likely to lose their concentration and being unable to notice the infrequent sign happened in the crowd. This emphasises on the need of automating the crowd monitoring with the support of computers. The next section will briefly discuss the trends of the related works that applied the computer-based approach into crowd monitoring.

\subsection{Current Approaches in Crowd Monitoring}
The state-of-the-art in crowd monitoring focused on three directions using computer vision, sensory data analysis and social media analysis. Most related works in the computer vision based approach applied the image processing techniques to analyse the video recorded from the CCTV cameras. One of the weakness of this technique was that the camera were not always located in the best location for crowd monitoring purpose \citep{Davies1995}. Furthermore, the performance of this technique was heavily affected by obstacles and low lighting condition \citep{Wirz2012}

Sensory data analysis approach relies on the use of sensing techniques using sensors built on mobile phones, such as GPS receivers \citep{Wirz2012} and accelerometers \citep{Roggen2011} to monitor the condition of the crowd. This approach is not limited by the field of view like in computer vision based approach, thus it can produce the global situation about a crowd\citep{Wirz2012}. However, there was a need to install special software or application, on the devices of the participants. The monitoring systems also required access to the data collected from the sensors on participants' mobile devices as well. This might lead to the issues regarding privacy from the users.

A special type of sensor, sometimes known as the ``soft sensor'' \citep{Ramesh2014} that was used in crowd monitoring is the social network services (SNS). A recent work by \citep{DelirHaghighi2013} utilised social media as the information source for contextual data and applied sentiment analysis on collected data to identify the crowd types. As this method used the data generated by the users on the Internet, it did not require the hardware or software deployed at the venue.

Table \ref{table:summaryOfLitReview} summarises the state-of-the-art techniques in crowd monitoring and the information sources as well as their limitations. Another limitation is that most of those works did not use a standard model that can represent and distinguish different types of crowd.

\begin{table}[!htbp]
\centering
\caption{Summary of state-of-the-art crowd monitoring techniques}
\label{table:summaryOfLitReview}
\begin{tabular}{|p{3cm}|p{5cm}|p{6cm}|}

\hline
\textbf{Technique} 								& \textbf{Information sources} 								& \textbf{Limitation} \\ \hline \hline
\multirow{3}{\linewidth}{Computer vision} 		& CCTV record \citep{Davies1995} 							& Not located in the best location for monitoring \\
 												& \multirow{2}{\linewidth}{Thermal video \citep{Pham2007}} 	& Affected by low lighting condition and obstacles \\
												& 															& Feasibility deploying in open area \\ \hline
\multirow{2}{\linewidth}{Sensory data analysis}	& GPS \citep{Wirz2012} 										& Require installing software \\
												& Accelerometer \citep{Roggen2011}							& Require access to data captured by sensors \\
												&															& Privacy concern \\ \hline
Social media analysis							& Twitter \citep{DelirHaghighi2013}							& Potential but very limited work has been done \\ \hline									
\end{tabular}
\end{table}

Regarding crowd modelling, \citet{Berlonghi1995}'s definition of eleven crowd types was one of the most widely adopted work in emergency management literature \citep{FEMA2005, EMA1999}. Each crowd type was described by the purpose of gathering and the activities in the crowd. However, using only these definitions, it is very difficult for a computer-based approach to distinguish different crowd types because Berlonghi's model lacked the features or attributes to make the distinction of crowd types. Another notable work on the classification of crowd was from \citet{Lofland1985} who categorised the crowd by the motivating emotions: \textit{anger}, \textit{fear} and \textit{joy}. This classification was based on the studies on collective behaviour and emotions \citep{Lofland1985,Smelser1998,Brown1954}. These studies suggests that emotion can be used as a feature to distinguish different crowds. In this case, a mapping between a crowd type and the associated emotions is desirable. 

\section{Research Questions}
Related studies have shown the need of crowd monitoring for emergency management in mass gathering. Although different approaches have been proposed to support the crowd monitoring, they had both strength and limitation. This research is proposed to answer these three questions:
\begin{enumerate}
\item How social media can be used as the information sources for context data used in crowd monitoring?
\item How to incorporate a standard crowd model to represent different types of crowd in crowd monitoring?
\item How emotion can be used as the feature to enable the automatic classification of crowd types?
\end{enumerate}

\section{Thesis Structure}
The rest of this thesis will be structured as follow. Chapter \ref{ch:litReview} will discuss and analyse the state-of-the-art in crowd monitoring and existing work in crowd modelling. Chapter \ref{ch:approach} proposes our crowd monitoring framework using the emotion analysis of social media. Chapter \ref{ch:eval} introduce our implementation and evaluation of the proposed framework using a case study with historical data. The thesis ends with Chapter \ref{ch:conclusion} which summarises our research contributions and introduce the potential directions for the future works.

\section{Conclusion}
This chapter has introduced the motivation of the research as well as a brief overview on the background of crowd monitoring for emergency management in mass gathering. This chapter also outlines the potentials and gaps regarding the related works, which our research aims to address.