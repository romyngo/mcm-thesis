As concluded in Chapter \ref{ch:litReview}, with the rapid development of mobile devices and mobile computing, the literature review has showed the potential of combining different information sources, such as mobile sensors and social media, in a crowd monitoring approach. In our framework, the context data layer is designed to fulfill this objective and serves as the input for the other layers down the stack.

The context data layer collects information about the context, which is the mass gathering event. In the development of a domain ontology for mass gathering, \citet{DelirHaghighi2013a} has summarised a set of related features that might have the effect on the safety of an event, such as environmental factors, event type, crowd size and venue. These knowledge can be retrieved from different sources. For this crowd monitoring framework, we propose three main sources: 
\begin{inparaenum}[i)]
\item event information;
\item mobile sensing;
\item social media
\end{inparaenum}, each of which will be discussed in detail in the following sections.

\subsection{Event Information}
Event information includes information about the venue, expected number of participants or crowd size, and the type of event. This information is static and not required to be collected in real time, hence it can be prepared in the pre-event phase of the emergency management. Ideally, it can be either crawled from the events' homepages or provided by the event organizers and gathered at a centralized system for the monitoring phase.

One of the important features of a mass gathering in the context of emergency management is crowd density, which is the key factor leading to trampling and crushing incidents \citet{Lee2005}. This information can be estimated by using collected data above, namely the estimated crowd size, the venue capacity and the size of venue following below formula.

\[
crowd\_density = crowd\_size / venue\_size
\]
where, \(crowd\_density\) is the estimated crowd density, \(crowd\_size\) is the number of participants or the venue capacity and \(venue\_size\) is the area of the venue holding the event.

\subsection{Mobile Sensing}

Mobile sensing paradigm has become an emerging topic among recent studies in mobile and context aware computing. It leverages the power of sensor-enhanced mobile devices to acquire information about the context \citep{guo2014participatory}. The mobile devices include modern smart-phones and devices, which are integrated with a wide range of sensors such as GPS receivers, accelerometers and ambient light sensors.

Using the built-in GPS receiver in mobile phones to collect information for crowd monitoring has been proposed in a related works proposed by \citet{Wirz2012}. As mentioned in the literature review, in this study, GPS is used to obtain the current location of the phone bearer. The continuous sampling of the location can be used to estimate the movement of the crowd and the crowd density.

Another integrated sensor in mobile phones that can be used to gather context data is the accelerometer. Accelerometer is often used for the purpose of activity recognition \citep{ravi2005activity, kwapisz2011activity}. In our context of crowd monitoring, the literature review has noted a related work from \citet{Roggen2011}, where several crowd activities can be determined by analysing the pattern of accelerometer's data.

Interestingly, the microphone in mobile phones can also act as a sensor to count the number of devices in a crowd \citep{Kannan2012, Xu2013}, thus can be eventually used to roughly estimate the number of participants. Table \ref{table:mobileSensingCrowdFeature} summarises the possible sensing sources that can be integrated into our framework and the knowledge about a crowd that can be inferred from these sensors.

\begin{table}
\caption{Mobile sensors and Crowd features}
\label{table:mobileSensingCrowdFeature}
\centering
\begin{tabular}{|l|l|}
\hline
\textbf{Mobile sensor} & \textbf{Crowd features} \\
\hline
GPS receiver & Movement and density \\
\hline
Accelerometer & Activity \\
\hline
Microphone & Density \\
\hline
\end{tabular}
\end{table}

The role of mobile sensing in our crowd monitoring framework is to provide the real-time data about a mass gathering event. As mentioned in the previous chapter, the during event phase in emergency management requires real-time decision making, which requires continuously updated intelligence from the crowd monitoring system. The reason for this requirement is that, according to \citet{Berlonghi1995}, a particular crowd can change from one type to another type during the event. This dynamic nature of a crowd is the key factor that we want to address in our approach by integrating the mobile sensing as one of the sources of context data.

\subsection{Social Media}

Social media is another source of information that is capable of providing real-time data. In a related work, social media is considered as a special ``soft sensor'' in the mobile sensing techniques used in crowd monitoring \citep{Ramesh2014}. Social media is a generic term referring to a wide range of Internet-based tools that enable users to create and share content. These tools include social network sites, such as Twitter and Facebook, Internet forums and channels. Among those tools, Twitter is by far the most commonly utilized social media in research because of the huge volume of user base and the availability of the API which makes the data highly accessible.

Studies show that during emergency situation, there is significant use of social media to report about the incident. One of the most well known research is the potential of earthquake detection by Twitter by \citet{Sakaki2010}. In the context of crowd monitoring, \citet{DelirHaghighi2013} has also proposed an approach to analyse tweets and capture the bipolar crowd mood. This has proven the feasibility of probing the social media to detect emergency situation.

In our proposed framework, social media is also integrated as one of the input for contextual data. If mobile sensing mentioned in the previous section can be considered as a pro-active mechanism where data is continuously sampling and analysed to detect critical crowd condition, social media provides us a reactive channel where we can capture a crowd incident as soon as it is reported by the Internet user. 

\section{Crowd Model}

The next layer in the crowd monitoring framework is the crowd model, which consists of a crowd typology and a set of crowd attributes for classification a crowd into a specific type.

\subsection{Comparison of different Crowd Models}
As mentioned in the literature review, there has been a very limited work on crowd modelling. From the perspective of emergency planning, Berlonghi's model is among the most widely adopted by emergency management bureaus worldwide \citep{FEMA2005, EMA1999}. Our literature also highlights several notable works in other disciplines that attempt to classify and describe different crowd types. Interestingly, in spite of having different disciplinary views, there are similarities and overlaps can be observed between those works and the model proposed by Berlonghi. Berlonghi has proposed the most number of crowd types and his eleven crowd types can be mapped into the crowd types mentioned in other studies and vice versa. Hence, in our approach, we will employ Berlonghi's model as the baseline to make distinction of different crowds.

Table \ref{table:crowdModelComparison} presents our chosen crowd types based on Berlonghi's work and compares each crowd type with the crowd types defined in the related works. From these related works, the definition and description of each crowd type are collected and gathered to construct a more detailed explanation of the crowd type.

\subsection{Crowd Types}
Explain each type
\begin{itemize}
\item Ambulatory crowd
\item Limited movement crowd
\item Crowd of spectators
\item Participatory crowd
\item Expressive crowd
\item Aggressive crowd
\item Crowd of demonstrator
\item Escaping crowd
\item Looting crowd
\item Rushing crowd
\item Violent crowd
\end{itemize}

\subsection{Crowd Attributes}
Explain why we need to add attribute. because the definition is not sufficient to classify. need human judgement. the need for a systematic approach.

Discuss each attribute
\begin{itemize}
\item Level of Density
\item Level of Movement
\item Crowd Activities
\item Motivating Emotions
\end{itemize}

Table showing the crowd type and the attributes

Among those attributes, we focus on the motivating emotions and discuss further in the subsections

\subsubsection{Human Basic Emotion}
List work on human basic emotions
8 emotions
6 emotions
4 emotions
\subsubsection{Emotion and Collective Behaviour}
\citet{Lofland1985} and \citet{Smelser1998} identify joy, fear and anger can motivate the formation of collective behaviour. 

\subsubsection{Mapping from Emotions to Crowd Types}
Insert the table here

\subsubsection{Social Media Analysis and Emotion Capture}
By probing social media, we can capture the emotion

\section{Crowd Monitoring}