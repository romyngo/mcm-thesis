% ************************** Thesis Abstract *****************************
% Use `abstract' as an option in the document class to print only the titlepage and the abstract.
\begin{abstract}

In emergency management for mass gathering, crowd monitoring is playing a significant role to provide timely response and effective resource allocation. Much effort has been done to introduce a computer based approach in crowd monitoring using different information sources and analysis techniques. One of the potential sources of information is social media because they do not require hardware or software pre-installed, however very limited work has been done using social media in crowd monitoring. Most of the existing approaches have a limitation that is the lack of a standard crowd model which can enable the consistency and interoperability between different monitoring systems by providing an uniform classification of different crowd types.

This research focuses on the use of social media as an information source for crowd monitoring and a standard crowd model to represent and distinguish different types of crowds. The effect of emotion on crowd behaviour is also taken into consideration. As a result, a crowd monitoring framework is proposed using the emotion analysis of social media to identify the crowd types in a event. An experiment with a case study using historical data of a past event has showed the proposed framework can detect the correct crowd types in a timely manner.

Future work is recommended to integrate more information sources and analysis techniques into the framework to provide additional contextual information to further distinguish the crowd types under the same emotional states, for example, using activity recognition technique over the data collected from accelerometers built on the mobile phones of participants.

\end{abstract}
